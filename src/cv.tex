%%%%%%%%%%%%%%%%%%%%%%%%%%%%%%%%%%%%%%%%%
% "ModernCV" CV and Cover Letter
% LaTeX Template
% Version 1.1 (9/12/12)
%
% This template has been downloaded from:
% http://www.LaTeXTemplates.com
%
% Original author:
% Xavier Danaux (xdanaux@gmail.com)
%
% License:
% CC BY-NC-SA 3.0 (http://creativecommons.org/licenses/by-nc-sa/3.0/)
%
% Important note:
% This template requires the moderncv.cls and .sty files to be in the same 
% directory as this .tex file. These files provide the resume style and themes 
% used for structuring the document.
%
%%%%%%%%%%%%%%%%%%%%%%%%%%%%%%%%%%%%%%%%%

%----------------------------------------------------------------------------------------
%	PACKAGES AND OTHER DOCUMENT CONFIGURATIONS
%----------------------------------------------------------------------------------------

\documentclass[1pt,a4paper,verdana]{moderncv} % Font sizes: 10, 11, or 12; paper sizes: a4paper, letterpaper, a5paper, legalpaper, executivepaper or landscape; font families: sans or roman



\moderncvstyle{casual} % CV theme - options include: 'casual' (default), 'classic', 'oldstyle' and 'banking'
\moderncvcolor{blue} % CV color - options include: 'blue' (default), 'orange', 'green', 'red', 'purple', 'grey' and 'black'



\usepackage{lipsum} % Used for inserting dummy 'Lorem ipsum' text into the template

\usepackage{xcolor}
\usepackage[scale=0.86]{geometry} % Reduce document margins
%\setlength{\hintscolumnwidth}{3cm} % Uncomment to change the width of the dates column
%\setlength{\makecvtitlenamewidth}{10cm} % For the 'classic' style, uncomment to adjust the width of the space allocated to your name

%----------------------------------------------------------------------------------------
%	NAME AND CONTACT INFORMATION SECTION
%----------------------------------------------------------------------------------------

\firstname{Egor} % Your first name
\familyname{Volkov} % Your last name

% All information in this block is optional, comment out any lines you don't need



\mobile{+7 (977) 882-23-79}
\email{ega.volk@gmail.com}
 % The first argument is the url for the clickable link, the second argument is the url displayed in the template - this allows special characters to be displayed such as the tilde in this example
%\extrainfo{additional information}
 % The first bracket is the picture height, the second is the thickness of the frame around the picture (0pt for no frame)
%\quote{"A witty and playful quotation" - John Smith}

%----------------------------------------------------------------------------------------

\begin{document}

\makecvtitle % Print the CV title
%----------------------------------------------------------------------------------------
%	EDUCATION SECTION
%----------------------------------------------------------------------------------------
\section{Education}
\cventry{09.2017 - Present}{Bachelor Applied mathematics and Computer science}{Moscow Institute of Physics and Technology (State University)}{Department of innovations and high technologies}{Moscow}{\textit{CGPA -- 8.18(out of 10)} upto 1st October 2019.}
%\cventry{09.2015 - 06.2017}{High School}{General Lyceum "AMTEK"} {Cherepovets}{}{High school Certificate with honors and Medal for special achievements in learning. \newline\textit{CGPA -- 5(out of 5)}}
%\cventry{09.2010 - 06.2015}{Secondary School}{General Lyceum "AMTEK"} {Cherepovets}{}{Secondary school Certificate with honors. \textit{CGPA -- 5(out of 5)}}
%\cventry{09.2006 - 06.2010}{Primary School}{School 39}{Cherepovets}{}{}
\cventry{09.2006 - 06.2017}{School}{General Lyceum "AMTEK"} {}{Cherepovets}{School Certificate with honors and Medal for special achievements in learning. \newline \textit{CGPA -- 5(out of 5)}.}
%----------------------------------------------------------------------------------------
%	ADDITIONAL EDUCATION SECTION
%----------------------------------------------------------------------------------------
\section{Additional education}
\cventry{09.2019 - Present}{Specialization Machine learning and data analysis}{\href{https://www.coursera.org/specializations/machine-learning-data-analysis}{\textcolor{black!65}{\underline{Coursera}}}}{}{}{At present, completed \textit{\href{https://www.coursera.org/account/accomplishments/verify/U3T7FHLECV4A}{\textcolor{black!65}{\underline{Math and Python for Data Analysis}}}} course.}
\cventry{2016}{Educational center "Sirius"}{Mathematics course in January; Big Data course in July}{}{Sochi}{I made a simple music recommendation system.}
\cventry{2015, 2016}{Summer informatics school}{Level B in 2015, Level A' in 2016}{}{Kostroma}{}
%----------------------------------------------------------------------------------------
%	EXPERIENCE SECTION
%----------------------------------------------------------------------------------------
\section{Experience}
\cventry{07.2019 - 10.2019}{Software engineer intern}{Yandex}{}{Moscow}{C++ and Python backend development. I worked in \textit{\href{https://market.yandex.ru}{\textcolor{black!65}{\underline{Yandex.Market}}}}. In particular, I was improving the search infrastructure and made some features for monetization.}
\cventry{07.2020 - 08.2020}{Data analytics intern}{JetBrains}{}{Saint Petersburg}{I've been clustering users based on the technologies they use.}
%----------------------------------------------------------------------------------------
%	PROJECTS
%----------------------------------------------------------------------------------------
\section{Projects}
\cventry{2018}{Sport stats telegram bot}{}{}{} {This bot parses  \textit{\href{https://sportbox.ru}{\textcolor{black!65}{\underline{sports website}}}} and provides results to the user. Code is \textit{\href{https://github.com/egavolk/MIPT_Python/tree/master/sport_stats_bot}{\textcolor{black!65}{\underline{here}}}.}}
\cventry{2017 - 2019}{Educational projects}{}{}{} {I have implemented different algorithms and data structures. Among them suffix tree and suffix automaton, maximum flow problem in $O(V^3)$ complexity, fast next permutation on a subsegment using treap, determining dots in a nonconvex polygon using scanning line, std-like tuple implementation and others. Code is \textit{\href{https://github.com/egavolk/MIPT_CppAADS}{\textcolor{black!65}{\underline{here}}}}.}
%----------------------------------------------------------------------------------------
%	AWARDS SECTION
%----------------------------------------------------------------------------------------
\section{Awards}
\cventry{2018}{ICPC, Moscow Programming Contest}{}{}{}{First Stage Award Winner.}
\cventry{2017}{Russian olympiad in informatics}{}{}{}{Winner of Regional Stage. Final Stage participant.}
\cventry{2015 - 2017}{School olympiads}{}{}{}{I have participated in many All-Russian and Regional olympiads in mathematics and informatics. I have become a winner or prizewinner in some of them.}
%----------------------------------------------------------------------------------------
%	SKILLS SECTION
%----------------------------------------------------------------------------------------
\section{Skills}
\cvitem{Computer skills}{Machine learning, Algorithms and Data Structures, \textsc{C++}, \textsc{Python}, \textsc{Java}, Linux, Microsoft Windows, SQL, Multithreaded synchronization, Git, Design patterns and refactoring.}
\cvitem{Math skills}{Probability theory, Math statistics, Linear algebra, Calculus, Discrete analysis, Combinatorics and number theory.}
\cvitem{GitHub}{\url{https://github.com/egavolk}. You can find here all previously referenced code.}
\cvitem{Languages}{English, Russian.}


\end{document}
